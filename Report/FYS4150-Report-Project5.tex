\documentclass{article}

\usepackage[margin=0.5in,bottom=1in,footnotesep=1in]{geometry}

\usepackage{amsmath}


\usepackage{multicol}
\setlength{\columnsep}{1cm}
\usepackage[]{algorithm2e}

\usepackage{lipsum}% for dummy text
\usepackage[varg]{txfonts}
\usepackage{graphicx}
\usepackage{subcaption}
\usepackage{multirow}

\usepackage[font=small,labelfont={sf,bf}]{caption}

\usepackage{color}

\usepackage[export]{adjustbox}

\usepackage{titlesec}
\titleformat{\section}{\fontfamily{phv}\fontsize{12}{15}\bfseries}{\thesection}{1em}{}
\titleformat{\subsection}{\fontfamily{phv}\fontsize{10}{15}\itshape}{\thesubsection}{1em}{}
\titleformat{\subsubsection}{\fontfamily{phv}\fontsize{9}{15}\bfseries}{\thesubsubsection}{1em}{}


\title{\textbf{FYS4150 Project 5: \\$N$-body simulation of an open galactic cluster}}
\author{Marie Foss (\# 56), Maria Hammerstr{{\o}}m (\# 59)}
\date{}

\begin{document}

\maketitle

\begin{abstract}
	\noindent \lipsum[1]
	\vspace*{2ex}
	
	\noindent \textbf{Github:} \textit{https://github.com/mariahammerstrom/Project5}
	\vspace*{2ex}
\end{abstract}



\begin{multicols}{2}

\section{Introduction}

An open cluster is a group of up to a few thousand stars that are gravitationally bound to each other. These stars are created from the same giant molecular cloud, 

These stars have roughly the same age and are created from the same giant molecular cloud. Open clusters are interesting subjects in the study of stellar evolution, because of the member's similar age and composition, making their properties such as distance, age, metallicity and extinction more easy to determine. 

Once an open cluster has been formed, it will gradually dissipate as its members get ejected from the cluster due to random collisions, lasting only a few hundred million years.

Open clusters are usually found in the arms of spiral galaxies or in irregular galaxies. 

(Images which shows an open cluster, and a galaxy with visible clusters - HII regions.)

In this project we looked at a simple model for how an open cluster is made from a gravitational collapse and for the interactions among a large number of stars. We have simulated a "cold collapse", meaning the particles will have little or no initial velocity. 



...




\section{Methods}

The numerical algorithms we wish to implement and compare are the fourth-order Runge-Kutta method and the Velocity-Verlet method.

...




\section{Results}

...


\section{Conclusions}

...




\section{List of codes}

The codes developed and used for this project are:\

...

\end{multicols}

\end{document}
